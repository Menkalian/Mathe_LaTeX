%! Suppress = LineBreak
\documentclass{scrreprt}

% Spracheinstellungen
\usepackage[utf8]{inputenc}
\usepackage[T1]{fontenc}
\usepackage[english, ngerman]{babel}
\usepackage{amsmath}
\usepackage{amsfonts}

\title{Mathe\_Übungen\_Abgabe} % Wir können ja auch immer nur die Lösung exportieren.
\author{Inf19 Gruppe B4}

\begin{document}

    \maketitle

    % Beginn einer neuen Übung
    \newpage
    \pagenumbering{arabic}
    \setcounter{page}{1}

    \section*{Lösung Übung 1}
    \begin{enumerate}
        \item[Aufgabe 1] {
            \begin{enumerate}
                \item[a)]{
                    $\langle a_n \rangle = \langle \frac{10^6}{n}\rangle$\newline
                    $\lim \limits_{n \to \infty} a_n =
                    \lim \limits_{n \to \infty} \frac{10^6}{n} =
                    \lim \limits_{n \to \infty} (10^6 \cdot \frac{1}{n} ) =
                    10^6 \cdot \lim \limits_{n \to \infty} (\frac{1}{n}) =
                    10^6 \cdot 0 = \underline{0}$
                }
                \item[b)]{
                    $\langle a_n \rangle = \langle \frac{\sqrt{n}}{n}\rangle$\newline
                    $\lim \limits_{n \to \infty} a_n =
                    \lim \limits_{n \to \infty} \frac{\sqrt{n}}{n} =
                    \lim \limits_{n \to \infty} \frac{1}{\sqrt{n}} =
                    \underline{0}$\\
                    da $\sqrt(n)$ über alle Grenzen wächst.
                }
                \item[c)]{
                    $\langle a_n \rangle =\langle (-\frac{3}{5})^n + 1\rangle$\newline
                    $\lim \limits_{n \to \infty} a_n =
                    \lim \limits_{n \to \infty} [(-\frac{3}{5})^n + 1] =
                    \lim \limits_{n \to \infty} (-\frac{3}{5})^n + \lim \limits_{n \to \infty} 1 =
                    0 + 1 = \underline{1}$
                }
                \item[d)]{
                    $\langle a_n \rangle = \langle\cos(n\cdot\frac{\pi}{2})\rangle$\newline
                    Konvergiert die Folge?\newline
                    $\Rightarrow$ Nutzung Theorem: Eine Folge konvergiert genau dann, wenn sie beschränkt und monoton ist.\newline

                    Prüfung der Schranken (trivial für unmodulierten Cosinus):\newline
                    \textbullet $\sup a_n = 1$\newline
                    \textbullet $\inf a_n = -1$\newline

                    Prüfung der Monotonie:\newline
                    Fälle der Werte in der Folge:
                    \begin{itemize}
                        \item{$n\mod 4 = 0 \rightarrow \cos(4k\cdot \frac{\pi}{2}) = 1 ; k\in \mathbb{N} $}
                        \item{$n\mod 4 = 1 \text{ od. } 3 \rightarrow \cos((2k + 1)\cdot \frac{\pi}{2}) = 1 ; k\in \mathbb{N} $}
                        \item{$n\mod 4 = 2 \rightarrow \cos((4k + 2)\cdot \frac{\pi}{2}) = -1 ; k\in \mathbb{N} $}
                    \end{itemize}
                    $\Rightarrow$ \textbf{KEINE} Monotonie vorhanden\\
                    $\Rightarrow$ Folge konvergiert nicht und es gibt keinen Grenzwert.
                }
            \end{enumerate}
        }
        \newpage
        \item[Aufgabe 2] {
            \begin{enumerate}
                \item[a)]{
                    $\langle a_n\rangle = \langle \frac{2\cdot n^2 - n + 1}{n^2 +1}\rangle$\\
                    $\lim \limits_{n \to \infty} a_n =
                    \lim \limits_{n \to \infty} \frac{2\cdot n^2 - n + 1}{n^2 +1} =
                    \lim \limits_{n \to \infty} \frac{2 - \frac{1}{n} + \frac{1}{n^2}}{1+ \frac{1}{n^2}} =
                    \frac{2}{1} = \underline{2}$
                }
                \item[b)]{
                    $\langle a_n\rangle = \langle \frac{n-1}{n\cdot\sqrt{n}}\rangle$\\
                    $\lim \limits_{n \to \infty} a_n =
                    \lim \limits_{n \to \infty} \frac{n-1}{n\cdot\sqrt{n}} =
                    \lim \limits_{n \to \infty} \frac{\frac{1}{\sqrt{n}} - \frac{1}{n\cdot\sqrt{n}}}{1} =
                    \underline{0}$\\
                    da alle Brüche im Zähler den Grenzwert 0 haben (Permanenzeigenschaften bzgl. Potenzen und Wurzeln -> Zurückführbar auf $\frac{1}{n}$)
                }
            \end{enumerate}
        }
        \newpage
        \item[Aufgabe 3]{
            $\langle a_n\rangle = \langle \frac{(n-1)^2}{2\cdot n^2 + 1}\rangle$\\
            $\langle b_n\rangle = \langle \frac{1-n}{3\cdot n + 1}\rangle$\\
            $\lim \limits_{n \to \infty} a_n =
            \lim \limits_{n \to \infty} \frac{(n-1)^2}{2\cdot n^2 + 1} =
            \lim \limits_{n \to \infty} \frac{n^2-2n+1}{2\cdot n^2 + 1} =
            \lim \limits_{n \to \infty} \frac{1-\frac{2}{n}+\frac{1}{n^2}}{2 + \frac{1}{n^2}} =
            \underline{\frac{1}{2}}$\\
            $\lim \limits_{n \to \infty} b_n =
            \lim \limits_{n \to \infty} \frac{1-n}{3\cdot n + 1} =
            \lim \limits_{n \to \infty} \frac{-1+ \frac{1}{n}}{3 + \frac{1}{n}} =
            \underline{-\frac{1}{3}}$

            \begin{enumerate}
                \item [a)]{
                    $c_n = a_n + b_n$\\\\
                    $\lim \limits_{n \to \infty} c_n =
                    \lim \limits_{n \to \infty} a_n + \lim \limits_{n \to \infty} b_n =
                    \frac{1}{2}+(-\frac{1}{3})= \underline{\frac{1}{6}}$
                }
                \item [b)]{
                    $c_n = b_n - a_n$\\\\
                    $\lim \limits_{n \to \infty} c_n =
                    \lim \limits_{n \to \infty} b_n - \lim \limits_{n \to \infty} a_n =
                    -\frac{1}{3}-(\frac{1}{2})= \underline{-\frac{5}{6}}$
                }
                \item [c)]{
                    $c_n = a_n \cdot b_n$\\\\
                    $\lim \limits_{n \to \infty} c_n =
                    \lim \limits_{n \to \infty} a_n \cdot \lim \limits_{n \to \infty} b_n =
                    \frac{1}{2}\cdot(-\frac{1}{3})= \underline{-\frac{1}{6}}$
                }
                \item [d)]{
                    $c_n = \frac{a_n}{b_n}$\\\\
                    $\lim \limits_{n \to \infty} c_n =
                    \frac{\lim \limits_{n \to \infty} a_n} {\lim \limits_{n \to \infty} b_n} =
                    \frac {\frac{1}{2}}{-\frac{1}{3}}= \underline{-\frac{3}{2}}$\\
                    gültig für $n\geq 2$.
                }
                \item [e)]{
                    $c_n = \left\| b_n \right\|$\\\\
                    $\lim \limits_{n \to \infty} c_n =\\
                    \lim \limits_{n \to \infty} \left\| b_n \right\| =
                    \left\| \lim \limits_{n \to \infty} b_n \right\| =\\
                    \underline{\frac{1}{3}}$\\
                }
                \item [f)]{
                    $c_n = \sqrt{8\cdot a_n}$\\\\
                    $\lim \limits_{n \to \infty} c_n =
                    \lim \limits_{n \to \infty} \left( \sqrt{8} \cdot \sqrt{a_n}\right) =\\
                    2\sqrt{2} \cdot \lim \limits_{n \to \infty} \sqrt{a_n} =
                    2\sqrt{2} \cdot \sqrt{ \lim \limits_{n \to \infty} a_n } =\\
                    2\sqrt{2} \cdot \sqrt{\frac{1}{2}} =
                    2\sqrt{2} \cdot \frac{1}{\sqrt{2}} =
                    \underline{2}$\\
                }
            \end{enumerate}
        }\newpage
        \item[Aufgabe 4]{
            \begin{enumerate}
                \item [a)]{
                    $\langle a_n \rangle = \langle \frac{n^2 + 10}{n+1} \rangle$\\
                    $\lim \limits_{n \to \infty} a_n =
                    \lim \limits_{n \to \infty} \frac{n^2 + 10}{n+1}$\\
                    reziproke Folge: $\left\| \frac{n+1}{n^2 +10} \right\|$\\
                    $\lim \limits_{n \to \infty} \left\| \frac{n+1}{n^2 +10} \right\| =
                    \left\| \lim \limits_{n \to \infty} \frac{n+1}{n^2 +10} \right\| =
                    \left\| \lim \limits_{n \to \infty} \frac{\frac{1}{n}+\frac{1}{n^2}}{1 + \frac{10}{n^2}} \right\| =
                    \left\| 0 \right\| =
                    \underline{0}$\\
                    $\rightarrow$ reziproke Folge ist Nullfolge $\rightarrow$ $a_n$ wächst über jede Grenze.
                }
                \item[b)]{
                    $\langle a_n \rangle = \langle \frac{b_q \cdot n^q + b_{q-1} \cdot n^{q-1} + \dots + b_0} {c_p \cdot n^p + c_{p-1} \cdot n^{p-1} + \dots + c_0} \rangle$\\
                    ($a_n$ stellt einen allgemeinen Bruch aus zwei Polynomen dar.)\\\\
                    Fallunterscheidung:
                    \begin{enumerate}
                        \item [1.Fall:] q<p: Nullfolge\\ s. Handschriftliche Bearbeitung %TODO: Nicht 4B einschicken.
                        \item [2.Fall:] q=p: $\frac{b_q}{c_p}$\\
                        $\lim \limits_{n \to \infty} a_n =
                        \lim \limits_{n \to \infty} \frac{b_q \cdot n^q + b_{q-1} \cdot n^{q-1} + \dots + b_0} {c_p \cdot n^p + c_{p-1} \cdot n^{p-1} + \dots + c_0} =
                        \lim \limits_{n \to \infty} \frac{b_q \cdot n^0 + b_{q-1} \cdot n^{-1} + \dots + b_0 \cdot n^{-q}} {c_q \cdot n^0 + c_{q-1} \cdot n^{-1} + \dots + c_0\cdot n^{-q}} =
                        \underline{\frac{b_q}{c_p}}$

                        \item [3.Fall:] q>p: Wächst über jede Grenze (Beweis siehe Fall 1 für reziproke Folge)
                    \end{enumerate}
                }
            \end{enumerate}
        }
    \end{enumerate}

    % Beginn einer neuen Übung
    \newpage
    \pagenumbering{arabic}
    \setcounter{page}{1}

    \section*{Lösung Übung 1 - Abgabe}
    \begin{enumerate}
        \item[Aufgabe 1] {
            \begin{enumerate}
                \item[a)]{
                    $\langle a_n \rangle = \langle \frac{10^6}{n}\rangle$\newline
                    $\lim \limits_{n \to \infty} a_n =
                    \lim \limits_{n \to \infty} \frac{10^6}{n} =
                    \lim \limits_{n \to \infty} (10^6 \cdot \frac{1}{n} ) =
                    10^6 \cdot \lim \limits_{n \to \infty} (\frac{1}{n}) =
                    10^6 \cdot 0 = \underline{0}$
                }
                \item[c)]{
                    $\langle a_n \rangle =\langle (-\frac{3}{5})^n + 1\rangle$\newline
                    $\lim \limits_{n \to \infty} a_n =
                    \lim \limits_{n \to \infty} [(-\frac{3}{5})^n + 1] =
                    \lim \limits_{n \to \infty} (-\frac{3}{5})^n + \lim \limits_{n \to \infty} 1 =
                    0 + 1 = \underline{1}$
                }
            \end{enumerate}
        }
        \item[Aufgabe 2] {
            \begin{enumerate}
                \item[a)]{
                    $\langle a_n\rangle = \langle \frac{2\cdot n^2 - n + 1}{n^2 +1}\rangle$\\
                    $\lim \limits_{n \to \infty} a_n =
                    \lim \limits_{n \to \infty} \frac{2\cdot n^2 - n + 1}{n^2 +1} =
                    \lim \limits_{n \to \infty} \frac{2 - \frac{1}{n} + \frac{1}{n^2}}{1+ \frac{1}{n^2}} =
                    \frac{2}{1} = \underline{2}$
                }
            \end{enumerate}
        }
        \item[Aufgabe 3]{
            $\langle a_n\rangle = \langle \frac{(n-1)^2}{2\cdot n^2 + 1}\rangle$\\
            $\langle b_n\rangle = \langle \frac{1-n}{3\cdot n + 1}\rangle$\\
            $\lim \limits_{n \to \infty} a_n =
            \lim \limits_{n \to \infty} \frac{(n-1)^2}{2\cdot n^2 + 1} =
            \lim \limits_{n \to \infty} \frac{n^2-2n+1}{2\cdot n^2 + 1} =
            \lim \limits_{n \to \infty} \frac{1-\frac{2}{n}+\frac{1}{n^2}}{2 + \frac{1}{n^2}} =
            \underline{\frac{1}{2}}$\\
            $\lim \limits_{n \to \infty} b_n =
            \lim \limits_{n \to \infty} \frac{1-n}{3\cdot n + 1} =
            \lim \limits_{n \to \infty} \frac{-1+ \frac{1}{n}}{3 + \frac{1}{n}} =
            \underline{-\frac{1}{3}}$

            \begin{enumerate}
                \item [a)]{
                    $c_n = a_n + b_n$\\\\
                    $\lim \limits_{n \to \infty} c_n =
                    \lim \limits_{n \to \infty} a_n + \lim \limits_{n \to \infty} b_n =
                    \frac{1}{2}+(-\frac{1}{3})= \underline{\frac{1}{6}}$
                }
                \item [b)]{
                    $c_n = b_n - a_n$\\\\
                    $\lim \limits_{n \to \infty} c_n =
                    \lim \limits_{n \to \infty} b_n - \lim \limits_{n \to \infty} a_n =
                    -\frac{1}{3}-(\frac{1}{2})= \underline{-\frac{5}{6}}$
                }
                \item [c)]{
                    $c_n = a_n \cdot b_n$\\\\
                    $\lim \limits_{n \to \infty} c_n =
                    \lim \limits_{n \to \infty} a_n \cdot \lim \limits_{n \to \infty} b_n =
                    \frac{1}{2}\cdot(-\frac{1}{3})= \underline{-\frac{1}{6}}$
                }
                \item [d)]{
                    $c_n = \frac{a_n}{b_n}$\\\\
                    $\lim \limits_{n \to \infty} c_n =
                    \frac{\lim \limits_{n \to \infty} a_n} {\lim \limits_{n \to \infty} b_n} =
                    \frac {\frac{1}{2}}{-\frac{1}{3}}= \underline{-\frac{3}{2}}$\\
                    gültig für $n\geq 2$.
                }
                \item [e)]{
                    $c_n = \left\| b_n \right\|$\\\\
                    $\lim \limits_{n \to \infty} c_n =\\
                    \lim \limits_{n \to \infty} \left\| b_n \right\| =
                    \left\| \lim \limits_{n \to \infty} b_n \right\| =\\
                    \underline{\frac{1}{3}}$\\
                }
                \item [f)]{
                    $c_n = \sqrt{8\cdot a_n}$\\\\
                    $\lim \limits_{n \to \infty} c_n =
                    \lim \limits_{n \to \infty} \left( \sqrt{8} \cdot \sqrt{a_n}\right) =\\
                    2\sqrt{2} \cdot \lim \limits_{n \to \infty} \sqrt{a_n} =
                    2\sqrt{2} \cdot \sqrt{ \lim \limits_{n \to \infty} a_n } =\\
                    2\sqrt{2} \cdot \sqrt{\frac{1}{2}} =
                    2\sqrt{2} \cdot \frac{1}{\sqrt{2}} =
                    \underline{2}$\\
                }
            \end{enumerate}
        }
        \item[Aufgabe 4]{
            \begin{enumerate}
                \item [a)]{
                    $\langle a_n \rangle = \langle \frac{n^2 + 10}{n+1} \rangle$\\
                    $\lim \limits_{n \to \infty} a_n =
                    \lim \limits_{n \to \infty} \frac{n^2 + 10}{n+1}$\\
                    reziproke Folge: $\left\| \frac{n+1}{n^2 +10} \right\|$\\
                    $\lim \limits_{n \to \infty} \left\| \frac{n+1}{n^2 +10} \right\| =
                    \left\| \lim \limits_{n \to \infty} \frac{n+1}{n^2 +10} \right\| =
                    \left\| \lim \limits_{n \to \infty} \frac{\frac{1}{n}+\frac{1}{n^2}}{1 + \frac{10}{n^2}} \right\| =
                    \left\| 0 \right\| =
                    \underline{0}$\\
                    $\rightarrow$ reziproke Folge ist Nullfolge $\rightarrow$ $a_n$ wächst über jede Grenze.
                }
            \end{enumerate}
        }
    \end{enumerate}


\end{document}
